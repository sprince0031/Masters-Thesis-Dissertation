
% this file is called up by thesis.tex
% content in this file will be fed into the main document

%: ----------------------- introduction file header -----------------------
\chapter{Introduction}

% the code below specifies where the figures are stored
\ifpdf
    \graphicspath{{1_introduction/figures/PNG/}{1_introduction/figures/PDF/}{1_introduction/figures/}}
\else
    \graphicspath{{1_introduction/figures/EPS/}{1_introduction/figures/}}
\fi

% ----------------------------------------------------------------------
%: ----------------------- introduction content ----------------------- 
% ----------------------------------------------------------------------

Breast cancer is one of the most prevalent cancer types among women worldwide \parencite{Sung2021GlobalCountries}, which requires proper diagnosis for planning effective treatment. As with any cancer type, early and accurate diagnosis is critical for improving patient outcomes. Given that cancer pathology is a manual and time-consuming process, machine learning (ML) techniques are making significant strides in the field of breast cancer diagnosis, offering promising avenues to complement traditional methods. A recent systematic review by \textcite{Nemade2022ATechniques} considered 162 research publications for the time period of 2015-2021 and investigated the use of machine intelligence techniques in breast cancer diagnosis. The review found that ML algorithms demonstrated promising results in tasks such as computer-aided diagnosis (CAD) using mammograms, ultrasound image analysis, and even gene expression profiling. 

One area of machine learning research that has exploded in recent years is generative AI or Foundational Models \parencite{Bommasani2021OnModels} with the introduction of Large Language Models (LLMs) such as ChatGPT \parencite{Brown2020LanguageLearners} \parencite{OpenAI2023GPT-4Report}, Gemini \parencite{GeminiTeam2023Gemini:Models}, Claude, etc., and image generation models such as Stable Diffusion \parencite{Rombach2021High-ResolutionModels}, DALL-E \parencite{Ramesh2021Zero-ShotGeneration}, Midjourney and more. Exploring new use cases for such ground breaking models is essential as they can lead to important discoveries especially in cancer research with the increasing adoption of digital pathology which opens up new avenues for the automation of diagnostic processes with increased accuracy. Whole Slide Imaging (WSI) is the complete digitization of whole tissue slides, and this has led to the application of various sophisticated computational techniques to histopathologic analysis \parencite{Aeffner2019IntroductionAssociation.} \parencite{Madabhushi2016ImageOpportunities.}.

\section{Motivation}

In this thesis, the main focus is on the Human epidermal growth factor receptor 2 (HER2). HER2 is a crucial biomarker in the pathology of breast cancer, and its over-expression is related to aggressive tumor behaviour and poor prognosis \parencite{Slamon1987HumanOncogene}. Accurate HER2 scoring is needed to determine eligibility for targeted therapies such as trastuzumab \parencite{Slamon2001UseHER2}. HER2 testing has been done traditionally through a manual HER2 score assignment from immunohistochemically (IHC) stained tissue sections. This procedure is not only time-consuming but also highly subjective, causing inter-observer variability \parencite{Wolff2013RecommendationsUpdate.}. Inevitably, the subjectivity brings forth discrepancies in diagnoses and, therefore, treatment decisions for the patients. Hence there is a need for methods that are more reliable, standardized and efficient. In the very least, a companion tool that pathologists can use to get an idea of the expected result and then verify would improve on the existing process. 

\section{Aims and Objectives} 

With the above stated problems with HER2 scoring established, this thesis investigates the application of Stable Diffusion \parencite{Rombach2021High-ResolutionModels} to generate IHC-stained WSI patches from corresponding H\&E WSI patches. Stable Diffusion has been proven to be good for high-quality, contextually accurate image generation, even better than Generative Adversarial Neural Netwoks (GANNs) that were considered to be state-of-the-art in generative models \parencite{Baranchuk2021Label-EfficientModels}. Hence Stable Diffusion is an appealing candidate for the task. The key to this is to fine-tune a pre-trained Stable Diffusion model with a custom dataset containing pairs of the H\&E and IHC WSI patches. 

\vspace{5 mm}

Thus the main research questions that are aimed to be addressed in this thesis are:

\begin{enumerate}

\item Is there a visual correlation between the H\&E stain and the HER2 score of a corresponding IHC stain that may not be apparent or fully understood by human pathologists?

\item Can the current state-of-the-art generative models achieve a better result than previous work in this area where they used GANNs?

\item Can Stable Diffusion be leveraged to also output a latent representation of the image generated as a means of scoring (classification) of the generated IHC stain image?
 

\end{enumerate}

\section{Research Contribution}

As conveyed in the research questions, the initially planned scope was to not only fine-tune a pre-trained diffusion model with H\&E images as additional input to a prompt that describes the IHC score; but also to device a way for the model to output the score along with the generated IHC image while only giving the H\&E image as input. While the first part of this goal was achieved using a ControlNet \parencite{Zhang2023AddingModels} for fine-tuning the model, the latter part turned out to be beyond the scope of this thesis due to time and resource constraints. Ideas for further research for the third research question above are detailed in the future scope subsection of chapter 5. Another pertinent research area that has not been covered by this research would be to understand the morphological features of the H\&E stain that the model uses to obtain an IHC stain by using explainable AI techniques. This would help pathologists better understand the correlation and potentially also improve their own manual scoring process.

\pagebreak

\section{Thesis Outline} 

The remaining chapters of this dissertation are as follows:

\emph{Chapter Two} is an introductory discussion to Stable Diffusion and ControlNets. It also provides a review of the research that tried using GANNs for IHC stain generation which is where the main research question for this thesis was derived from. 

\emph{Chapter Three} demonstrates the implementation of the ControlNet framework which is used to fine-tune a Stable Diffusion 2.1 model.

\emph{Chapter Four} presents the results and a discussion about gauging its performance including for multiple variations of models that were trained.

\emph{Chapter Five} concludes the research and suggests possible future works that can continue the research <FINISH THIS>

\vspace{5 mm}

A series of documents have been included in the Appendix section of this dissertation. These are:

\begin{itemize}
\item \emph{Appendix A} outlines . . .

\item \emph{Appendix B} presents . . .

\item \emph{Appendix C} includes . . .
\end{itemize} 

\vspace{5 mm}

% ----------------------------------------------------------------------



