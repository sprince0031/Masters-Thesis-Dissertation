
% this file is called up by thesis.tex
% content in this file will be fed into the main document

%: ----------------------- introduction file header -----------------------
\chapter{Introduction}

% the code below specifies where the figures are stored
\ifpdf
    \graphicspath{{1_introduction/figures/PNG/}{1_introduction/figures/PDF/}{1_introduction/figures/}}
\else
    \graphicspath{{1_introduction/figures/EPS/}{1_introduction/figures/}}
\fi

% ----------------------------------------------------------------------
%: ----------------------- introduction content ----------------------- 
% ----------------------------------------------------------------------

Breast cancer is one of the most prevalent cancer types among women worldwide \parencite{Sung2021GlobalCountries}, which requires proper diagnosis for planning effective treatment. As with any cancer type, early and accurate diagnosis is critical for improving patient outcomes. Given that cancer pathology is a manual and time-consuming process, machine learning (ML) techniques are making significant strides in the field of breast cancer diagnosis, offering promising avenues to complement traditional methods. A recent systematic review by \textcite{Nemade2022ATechniques} considered 162 research publications for the time period of 2015-2021 and investigated the use of machine intelligence techniques in breast cancer diagnosis. The review found that ML algorithms demonstrated promising results in tasks such as computer-aided diagnosis (CAD) using mammograms, ultrasound image analysis, and even gene expression profiling. 

Researching new applications for latest advancements in the machine learning field One area of machine learning research that has exploded in recent years is generative AI with the introduction of Large Language Models (LLMs) such as ChatGPT, Gemini, Claude, etc., and image generation models such as Stable Diffusion and DALL-E. 

\section{Motivation}

In this thesis, the main focus is on the Human epidermal growth factor receptor 2 (HER2). HER2 is a crucial biomarker in the pathology of breast cancer, and its over-expression is related to aggressive tumor behavior and poor prognosis \parencite{Slamon1987HumanOncogene}. Accurate HER2 scoring is needed in order to determine eligibility for targeted therapies such as trastuzumab \parencite{Slamon2001UseHER2}. HER2 testing has been done traditionally through a manual HER2 score assignment from immunohistochemically (IHC) stained tissue sections. This procedure is not only time-consuming but also highly subjective, causing inter-observer variability. Inevitably, the subjectivity brings forth discrepancies in diagnoses and, therefore, treatment decisions for the patients, hence the need for methods that are more reliable, standardized and efficient. In the very least, a companion tool that pathologists can use to get an idea of the expected result and then verify would improve on the existing process. 

The recent advent of digital pathology has indeed opened up new avenues for the automation of diagnostic processes with increased accuracy. Whole Slide Imaging (WSI) is the complete digitization of whole tissue slides, and this has led to the application of various sophisticated computational techniques to histopathologic analysis. However, there still remains a problem in Digital IHC staining from H\&E-stained slides; the latter being the standard in most pathology laboratories. This is an assumption-based digital transformation of the actual staining of H\&E-stained images into IHC, which has strong promise to eliminate manual labelling work, saving time and ensuring good diagnostic confidence and quality.

This thesis investigates the application of Stable Diffusion—an advanced deep learning method—to digitally transform H\&E WSIs into IHC-stained WSIs for subsequent HER2 scoring. Stable Diffusion is being noticed as a method for high-quality, contextually accurate image generation and hence is an appealing candidate for the task. The promise of the approach in one basic sense lies in its potential to improve the efficiency, accuracy, and consistency of HER2 scoring in pathology of breast cancer.

The present study aims to establish this with digital tools that will diminish the important dependence on the manual staining and scoring used in standard diagnostic practices, hence yielding more standardized and reproducible result data.

One of the proposed methods for this research is the usage of a pre-trained model from Stable Diffusion for digital IHC staining over H\&E WSIs. Nevertheless, fine-tuning the existing Stable Diffusion model for breast cancer pathology is indeed a very challenging task. Despite very intensive efforts, the technical barrier has been experienced in this adaptation of the pre-trained model towards delivery of satisfactory results by the research. This aspect places great emphasis on the complexity in using general-purpose deep learning models for specific medical imaging tasks. As results have not been realized and this study is ongoing, it contributes to the emerging works on digital pathology; it identifies the potential and challenges of using very recent state-of-the-art advanced machine learning methodologies in medical diagnostics. Such insights can guide future work in devising more effective methods for digital staining and automated HER2 scoring, which could consequently improve breast cancer diagnosis and patient outcomes. 

\parencite{Zhang2023AddingModels}

% here we declare a new section

\section{Aims and Objectives} 

The main question that this dissertation addresses is  
In order to address this question, this work focuses on the following issues in the context of :

\begin{itemize}

\item Firstly, . . .

\item Secondly, . . . 

\end{itemize}



% here we declare a new section
\section{Methodology} 

In order to address this question the following approach was taken. 
 
 
% here we declare a new section
\section{Research Contribution}

The primary research contribution is: 





\section{Thesis Outline} 

The remaining chapters of this dissertation are as follows:


\emph{Chapter Two} is an introductory discussion to . . .

\emph{Chapter Three} is a historical review of  . . .

\emph{Chapter Four} presents the  . . . 

\emph{Chapter Five} describes the . . .

\emph{Chapter Six} draws conclusions and evaluates the . . .  Lastly, it suggests possible future works. 

\vspace{5 mm}

A series of documents have been included in the Appendix section of this dissertation. These are:

\begin{itemize}
\item \emph{Appendix A} outlines . . .

\item \emph{Appendix B} presents . . .

\item \emph{Appendix C} includes . . .
\end{itemize} 

\vspace{5 mm}

Attached to this dissertation is a CD containing the following items:

\begin{itemize}
\item \emph{folder 1}: . . .

\item \emph{folder 2}: . . .

\end{itemize}


% ----------------------------------------------------------------------



