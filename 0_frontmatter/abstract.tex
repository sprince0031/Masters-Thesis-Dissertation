
% Thesis Abstract -----------------------------------------------------


%\begin{abstractslong}    %uncommenting this line, gives a different abstract heading
\begin{abstracts}        %this creates the heading for the abstract page

\textbf{Objective: }The purpose of this research is to explore the use of the generative AI technique, Stable Diffusion, for the creation of immunohistochemical (IHC) images 'digitally stained' from Hematoxylin and Eosin (H\&E) whole slide images(WSI) patches and classifying thereof to obtain a corresponding HER2 score. The purpose is to check if the approach provides an efficient, cost-effective, and consistent alternative to current standard practice which is known to be expensive, subjective and time-consuming for the traditional manual HER2 assessment procedure.

\textbf{Design:} We use the state-of-the-art Stable Diffusion technique implemented using a ControlNet, which is a generative AI model that allows for fine controls in the output of the generated images. 

\textbf{Results:} The fine-tuned ControlNet model generates IHC-stained image patches with the best model producing samples that have learnt a lot of the structural features of the slide. Evaluation metrics, both quantitative and qualitative are explored using comparison algorithms and the expert opinion of a pathologist.

\textbf{Conclusions:} In the present study, we show the feasibility to apply Stable Diffusion in conjunction with a ControlNet as a core technique in Digital Pathology and Stain Transfer. The use of Stable Diffusion may provide more standard and hence, reliable HER2 assessment. The study seems promising in advancing the future of AI generators in clinical digital pathology.

\textbf{Keywords:} Stable Diffusion, ControlNet, Digital Pathology, HER2 Scoring, Immunohistochemistry, Hematoxylin and Eosin, Whole Slide Images, Generative AI, Breast Cancer, Deep Learning.



\end{abstracts}
%\end{abstractlongs}


% ---------------------------------------------------------------------- 

