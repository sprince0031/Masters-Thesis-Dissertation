
% Thesis Abstract -----------------------------------------------------


%\begin{abstractslong}    %uncommenting this line, gives a different abstract heading
\begin{abstracts}        %this creates the heading for the abstract page

Immunohistochemistry (IHC) test is done to determine the intensity of gene expression of
the human epidermal growth factor receptor 2 (HER2) protein responsible for the growth of healthy breast cells. If this gene is over expressed in patients diagnosed with breast
cancer, it will lead to rapid spread of the tumor. However, tumors that are HER2-positive
can be treated with effective chemotherapy if identified early. Traditional manual HER2
assessments are expensive, prone to subjectivity from pathologists and time-consuming.
This study investigates the use of the generative AI technique, Stable Diffusion, for generating a "digitally stained" immunohistochemical (IHC) image from Hematoxylin and
Eosin (H&E) whole slide images (WSIs) and then classify them to obtain a corresponding
HER2 score. Earlier attempts to perform digital staining or style transfer used Generative
Adversarial Neural Networks (GANNs) which yielded mixed results at best.

Stable Diffusion is a recent technique and has been at the forefront of the generative AI
revolution for generating extremely detailed and hyper-realistic images from text prompts
(text-to-image) or from a reference image (image-to-image). The jump in quality is due to
this technique leveraging the concept of training a model to transform random noise to the desired image described. Here, the image-to-image method is used to ground the model in the source H&E image to then generate a corresponding IHC stained image.

The pre-trained open-sourced Stable Diffusion model, SDXL from Stability.ai is fine-tuned
with a dataset containing 4,870 registered image pairs of IHC and H&E stained WSIs from
the BCI dataset (Liu et al., n.d.). The accuracy is compared with the results obtained from
other algorithms that used GANNs to achieve the style transfer. A final subjective
validation by a pathologist is also done on the generated IHC stain images to test for
efficacy.



\end{abstracts}
%\end{abstractlongs}


% ---------------------------------------------------------------------- 

