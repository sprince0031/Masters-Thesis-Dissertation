
% Thesis Abstract -----------------------------------------------------


%\begin{abstractslong}    %uncommenting this line, gives a different abstract heading
\begin{abstracts}        %this creates the heading for the abstract page

\textbf{Objective: }The purpose of this research is to explore the use of the generative AI technique, Stable Diffusion, for the creation of immunohistochemical (IHC) images 'digitally stained' from Hematoxylin and Eosin (H\&E) whole slide images (WSI) and classifying thereof to obtain a corresponding HER2 score. The purpose is to check if the approach provides an efficient, cost-effective, and consistent alternative to current standard practice which is known to be expensive, subjective and time-consuming for the traditional manual HER2 assessment procedure.

\textbf{Design:} We use the state-of-the-art Stable Diffusion technique, which is a generative AI model tested to produce very detailed and realistic images. In this study, we implement the image-to-image approach of Stable Diffusion for the first time by pre-training the model on source H\&E images in order to generate corresponding IHC-stained images. We use their pre-trained model SDXL from Stability.ai and fine-tune it using our dataset consisting of 4,870 registered image pairs of IHC- and H\&E-stained WSIs coming from the BCI dataset.

\textbf{Setting:} The study was conducted inside a digital pathology framework, which puts into action datasets and open-source AI models available from public sources. Setting includes a computational environment for model training and fine-tuning. The final set of images generated is validated subjectively by a pathologist to check the usability of the images.

\textbf{Results:} The fine-tuned Stable Diffusion model generated IHC-stained images, which were later classified to obtain HER2 scores. For comparison, the accuracy of the digital staining and scoring processes was made to earlier studies using Generative Adversarial Neural Networks (GANs) for style transfer. Initial comparison showed that Stable Diffusion performs better in generating digital stains of high quality; however, final efficacy and accuracy needs to be validated subjectively by the pathologist.

\textbf{Conclusions:} In the present study, we show the feasibility to apply Stable Diffusion as a core technique in Digital Pathology, particularly for creation of IHC-stained images from H\&E WSIs enabling HER2 scoring in breast cancer diagnostics. The realized results are encouraging; however, future validation by practicing clinicians is necessary to fully assess practical applicability of this approach. The use of Stable Diffusion may provide more standard and hence, reliable HER2 assessment. It will improve the overall workflow with regard to the diagnosis and patient outcome in the treatment of breast cancer.

\textbf{Keywords:} Stable Diffusion, Digital Pathology, HER2 Scoring, Immunohistochemistry, Hematoxylin and Eosin, Whole Slide Images, Generative AI, Breast Cancer, Deep Learning,



\end{abstracts}
%\end{abstractlongs}


% ---------------------------------------------------------------------- 

